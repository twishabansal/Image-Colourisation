\documentclass[12pt]{beamer}

\title[Image-Colourization]{Image Colourisation Project}
\subtitle{Jun-Aug 2020 {\textbar} WTEF Project {\textbar} Deep Learning}
\author[Team-9]{Priyansi {\textbar} Sejal Gupta {\textbar} Twisha Bansal}
\date{September 2020}

\usetheme{metropolis}
\usepackage{xcolor}
\usepackage{graphicx}
\usepackage{hyperref}

\graphicspath{ {/mnt/c/Users/Sejal/Desktop/project/} }

\begin{document}

\begin{frame}
        \titlepage
\end{frame}

\begin{frame}[standout]
	\center{Our journey from being entirely clueless to completing an\\ Image Colourisation project\\ in Deep Learning}
\end{frame}

\begin{frame}{Objective}
	\begin{center}
		\includegraphics[width=11cm]{Objective}
	\end{center}
\end{frame}

\begin{frame}{Colorful Image Colorization paper by Richard Zhang, Phillip Isola, Alexei A. Efros}
	To hallucinate the most plausible colour version by training a CNN, to map from a grayscale input to a distribution over quantized colour value outputs\\
	\pause
	\textbf{Challenges}\\
	\begin{itemize}
		\item Advanced Mathematics
		\item Limited Resources
	\end{itemize}
\end{frame}

\begin{frame}[standout]
	\center{\alert{AutoEncoders}}\\
	A type of Neural Network used to learn representation for a set of data in an unsupervised manner
\end{frame}

\begin{frame}{CIELAB Colour Space}
	\begin{itemize}
		\item L channel: Lightness
		\item A channel: green to red
		\item B channel: blue to yellow
	\end{itemize}
\end{frame}

\begin{frame}{Technology Stack}
	\begin{columns}
		\begin{column}{0.5\textwidth}
			\begin{itemize}
				\item \textbf{Building the Model}\\
					- PyTorch\\
					- NumPy\\
					- Scikit-image\\
					- Matplotlib\\
			\end{itemize}
		\end{column}
		\begin{column}{0.5\textwidth}
			\begin{itemize}
				\item \textbf{Version Control}\\
					- Kaggle\\
				\item \textbf{Web App}\\
					- Streamlit\\
				\item \textbf{PaaS}\\
					- Heroku
			\end{itemize}
		\end{column}
	\end{columns}
\end{frame}

\begin{frame}{Model}
	\begin{center}
		\includegraphics[width=11cm]{AutoEncoder}
	\end{center}
\end{frame}

\begin{frame}{Model}
	\begin{itemize}
		\item Loss Function: \textbf{MSE Loss}
		\item Optimiser: \textbf{Adam}
		\item Range of Learning Rates used for training: \textbf{1e-3 - 1e-6}
	\end{itemize}
\end{frame}

\begin{frame}{Challenges}
	\begin{itemize}
		\item Lack of resources
			\pause
		\item MSE Loss is not a good parameter to measure human perception of colouring
	\end{itemize}
\end{frame}

\begin{frame}{Datasets}
	Trained on approximately \textbf{570K} images (Links provided below)
	\begin{itemize}
		\item \href{https://www.kaggle.com/lijiyu/imagenet}{ImageNet(50K Images)}\\
		\item \href{https://www.kaggle.com/hsankesara/flickr-image-dataset}{Flickr}\\
		\item \href{https://www.kaggle.com/huseynguliyev/landscape-classification}{Landscape Classification}\\
		\item \href{https://www.kaggle.com/moltean/fruits}{Fruits 360}\\
		\item \href{https://www.kaggle.com/chrisfilo/fruit-recognition}{Fruits Recognition}\\
		\item \href{https://www.kaggle.com/salil007/caavo}{Clothes Classification}\\
		\item \href{https://www.kaggle.com/jessicali9530/celeba-dataset}{CelebA Dataset}\\
		\item \href{https://www.kaggle.com/alessiocorrado99/animals10}{Animals 10}\\
		\item \href{https://www.kaggle.com/mistag/arthropod-taxonomy-orders-object-detection-dataset}{Arthropod Taxonomy Dataset}\\
	\end{itemize}
\end{frame}

\begin{frame}{Results: Good}
	\includegraphics[width=11cm]{pass-final}
\end{frame}

\begin{frame}{Results: Bad}
	\includegraphics[width=11cm]{fail}
\end{frame}

\begin{frame}{Web Application and Deployment}
	\textbf{Web App:}\\
	\begin{itemize}
		\item Made with Streamlit\\
		\item Why Streamlit:\\
			\begin{enumerate}
				\item Reducing app code to Python scripts
				\item Treating widgets like variables
				\item Reusing data with memoization
			\end{enumerate}
	\end{itemize}
	\textbf{Deployment:}
	\begin{itemize}
		\item Heroku as the cloud platform
	\end{itemize}
\end{frame}

\begin{frame}{Improvements}
	\begin{itemize}
		\item Adding more themes
		\item Automating the classification of themes
		\item Retrain with additional data generated using data augmentation
	\end{itemize}
\end{frame}

\begin{frame}{Sources}
	\begin{enumerate}
		\item Colorful Image Colorization paper by Richard Zhang, Phillip Isola, Alexei A. Efros: \textcolor{cyan}{\href{https://arxiv.org/pdf/1603.08511.pdf}{https://arxiv.org/pdf/1603.08511.pdf}}
		\item Applications of AutoEncoders - Image Colourisation: \textcolor{cyan}{\href{https://github.com/bnsreenu/python\_for\_microscopists}{https://github.com/bnsreenu/python\_for\_microscopists}}
	\end{enumerate}
\end{frame}

\begin{frame}{Our Project}
	\begin{itemize}
		\item Web Application: \textcolor{cyan}{\href{https://image-colouriser-streamlit.herokuapp.com/}{https://image-colouriser-streamlit.herokuapp.com/}}
		\item Gitlab: \textcolor{cyan}{\href{https://gitlab.com/twishabansal/image-colourisation}{https://gitlab.com/twishabansal/image-colourisation}}
		\item Kaggle Notebook: \textcolor{cyan}{\href{https://www.kaggle.com/sejalgupta01/image-colorization-starter}{https://www.kaggle.com/sejalgupta01/image-colorization-starter}}
	\end{itemize}
\end{frame}

\begin{frame}[standout]
	\center{Questions and Suggestions?}
\end{frame}

\end{document}
