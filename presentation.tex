\documentclass[16pt, subsection=false]{beamer}

\title{Colorful Image Colorization}
\subtitle{Jun-Jul 2020 WTEF Project using Deep Learning}
\author[Team 9]{Priyansi, Sejal, Twisha}
\date{24 June 2020}

\usetheme{Berlin}
\usecolortheme{seahorse}
\usefonttheme{structurebold}

\begin{document}

\section{Introduction}
\begin{frame}
	\titlepage
\end{frame}

\section{Overview}
\begin{frame}
	\frametitle{Overview}
	The objective of our project is to replicate the paper on Colorful Image Colorization, by Richard Zhang,
	Phillip Isola, Alexei A. Efros (5 October, 2016), and add the following tweaks:
	\begin{enumerate}
		\item Apply transfer learning
		\item Data augmentation
		\item Changing hyperparameters
		\item Expanding the dataset
	\end{enumerate}
	The workflow is as follows: \\
	Replicating the base model$\,\to\,$Improving it$\,\to\,$Making a web app$\,\to\,$Deploying it
\end{frame}

\section{Technology Stack}
\begin{frame}
	\frametitle{Technology Stack}
	\begin{enumerate}
		\item Deep Learning with PyTorch \\
		\item Flask
		\item Heroku
	\end{enumerate}
\end{frame}

\section{Description}
\begin{frame}
	\frametitle{Description}
	\begin{itemize}
		\item\textbf{The Objective}:``Given a grayscale photograph as input, the paper attacks the problem
			of hallucinating a \emph{plausible} color version of the photograph." - explains the Abstract.
			The project aims to achieve a``fully automatic approach that produces vibrant and realistic 
			colorizations."
		\item\textbf{How to Accomplish it}: The problem is dealt with as a classification task and uses class
			rebalancing at training time to increase the diversity of colors. The system is implemented as
			a feed-forward pass in a CNN at test time and is trained on over a million images.
	\end{itemize}
\end{frame}
\begin{frame}
	\frametitle{Description}
	\begin{itemize}
		\item\textbf{Why This Project}: Common approaches and solutions to tackle this problem either rely on
			significant user interaction or result in desaturated colorizations, in an effort to restore
			the ground image. Hence, this project, instead of making futile efforts to restore the
			original colorization, aims to produce the most plausible color version.
		\item\textbf{Evaluation and Success}: The algorithm used in the model is evaluated using the 
			\emph{``color Turing test"}, where human participants are asked to differentiate between the 
			ground image and the colorized image. The algorithm successfully fools humans in 32\% of the
			trials.
	\end{itemize}                                                                                                   \end{frame}

\section{Status}
\begin{frame}
	\frametitle{Status}
	The work on the project itself has not been started yet.
	All three of us are still in the learning phase.
	We plan to start working on it within the next two days.
\end{frame}

\end{document}
